% generated by GAPDoc2LaTeX from XML source (Frank Luebeck)
\documentclass[a4paper,11pt]{report}

\usepackage[top=37mm,bottom=37mm,left=27mm,right=27mm]{geometry}
\sloppy
\pagestyle{myheadings}
\usepackage{amssymb}
\usepackage[utf8]{inputenc}
\usepackage{makeidx}
\makeindex
\usepackage{color}
\definecolor{FireBrick}{rgb}{0.5812,0.0074,0.0083}
\definecolor{RoyalBlue}{rgb}{0.0236,0.0894,0.6179}
\definecolor{RoyalGreen}{rgb}{0.0236,0.6179,0.0894}
\definecolor{RoyalRed}{rgb}{0.6179,0.0236,0.0894}
\definecolor{LightBlue}{rgb}{0.8544,0.9511,1.0000}
\definecolor{Black}{rgb}{0.0,0.0,0.0}

\definecolor{linkColor}{rgb}{0.0,0.0,0.554}
\definecolor{citeColor}{rgb}{0.0,0.0,0.554}
\definecolor{fileColor}{rgb}{0.0,0.0,0.554}
\definecolor{urlColor}{rgb}{0.0,0.0,0.554}
\definecolor{promptColor}{rgb}{0.0,0.0,0.589}
\definecolor{brkpromptColor}{rgb}{0.589,0.0,0.0}
\definecolor{gapinputColor}{rgb}{0.589,0.0,0.0}
\definecolor{gapoutputColor}{rgb}{0.0,0.0,0.0}

%%  for a long time these were red and blue by default,
%%  now black, but keep variables to overwrite
\definecolor{FuncColor}{rgb}{0.0,0.0,0.0}
%% strange name because of pdflatex bug:
\definecolor{Chapter }{rgb}{0.0,0.0,0.0}
\definecolor{DarkOlive}{rgb}{0.1047,0.2412,0.0064}


\usepackage{fancyvrb}

\usepackage{mathptmx,helvet}
\usepackage[T1]{fontenc}
\usepackage{textcomp}


\usepackage[
            pdftex=true,
            bookmarks=true,        
            a4paper=true,
            pdftitle={Written with GAPDoc},
            pdfcreator={LaTeX with hyperref package / GAPDoc},
            colorlinks=true,
            backref=page,
            breaklinks=true,
            linkcolor=linkColor,
            citecolor=citeColor,
            filecolor=fileColor,
            urlcolor=urlColor,
            pdfpagemode={UseNone}, 
           ]{hyperref}

\newcommand{\maintitlesize}{\fontsize{50}{55}\selectfont}

% write page numbers to a .pnr log file for online help
\newwrite\pagenrlog
\immediate\openout\pagenrlog =\jobname.pnr
\immediate\write\pagenrlog{PAGENRS := [}
\newcommand{\logpage}[1]{\protect\write\pagenrlog{#1, \thepage,}}
%% were never documented, give conflicts with some additional packages

\newcommand{\GAP}{\textsf{GAP}}

%% nicer description environments, allows long labels
\usepackage{enumitem}
\setdescription{style=nextline}

%% depth of toc
\setcounter{tocdepth}{1}





%% command for ColorPrompt style examples
\newcommand{\gapprompt}[1]{\color{promptColor}{\bfseries #1}}
\newcommand{\gapbrkprompt}[1]{\color{brkpromptColor}{\bfseries #1}}
\newcommand{\gapinput}[1]{\color{gapinputColor}{#1}}


\begin{document}

\logpage{[ 0, 0, 0 ]}
\begin{titlepage}
\mbox{}\vfill

\begin{center}{\maintitlesize \textbf{ RepnDecomp \mbox{}}}\\
\vfill

\hypersetup{pdftitle= RepnDecomp }
\markright{\scriptsize \mbox{}\hfill  RepnDecomp  \hfill\mbox{}}
{\Huge \textbf{ Decompose representations of finite groups into irreducibles \mbox{}}}\\
\vfill

{\Huge  0.1 \mbox{}}\\[1cm]
{ 27 August 2018 \mbox{}}\\[1cm]
\mbox{}\\[2cm]
{\Large \textbf{ Kaashif Hymabaccus\\
    \mbox{}}}\\
\hypersetup{pdfauthor= Kaashif Hymabaccus\\
    }
\end{center}\vfill

\mbox{}\\
{\mbox{}\\
\small \noindent \textbf{ Kaashif Hymabaccus\\
    }  Email: \href{mailto://kaashif@kaashif.co.uk} {\texttt{kaashif@kaashif.co.uk}}\\
  Homepage: \href{https://kaashif.co.uk} {\texttt{https://kaashif.co.uk}}\\
  Address: \begin{minipage}[t]{8cm}\noindent
 TODO\\
 \end{minipage}
}\\
\end{titlepage}

\newpage\setcounter{page}{2}
\newpage

\def\contentsname{Contents\logpage{[ 0, 0, 1 ]}}

\tableofcontents
\newpage

     
\chapter{\textcolor{Chapter }{Block diagonalizing representations}}\label{Chapter_Block_diagonalizing_representations}
\logpage{[ 1, 0, 0 ]}
\hyperdef{L}{X7B6A137E7D92754D}{}
{
  
\section{\textcolor{Chapter }{Finding the correct basis}}\label{Chapter_Block_diagonalizing_representations_Section_Finding_the_correct_basis}
\logpage{[ 1, 1, 0 ]}
\hyperdef{L}{X7DEEF8F58622BFDC}{}
{
  Given a representation $\rho : G \to GL(V)$, it is often desirable to find a basis for $V$ that block diagonalizes each $\rho(g)$ with the block sizes being as small as possible. 

\subsection{\textcolor{Chapter }{BlockDiagonalBasis (for IsGroupHomomorphism)}}
\logpage{[ 1, 1, 1 ]}\nobreak
\hyperdef{L}{X796E97C0829CAC27}{}
{\noindent\textcolor{FuncColor}{$\triangleright$\enspace\texttt{BlockDiagonalBasis({\mdseries\slshape rho})\index{BlockDiagonalBasis@\texttt{BlockDiagonalBasis}!for IsGroupHomomorphism}
\label{BlockDiagonalBasis:for IsGroupHomomorphism}
}\hfill{\scriptsize (attribute)}}\\
\textbf{\indent Returns:\ }
Basis for $V$ that block diagonalizes $\rho$. 



 Let $G$ have irreducible representations $\rho_i$, with dimension $d_i$ and multiplicity $m_i$. The basis returned by this operation gives each $\rho(g)$ as a block diagonal matrix which has $m_i$ blocks of size $d_i \times d_i$ for each $i$. }

 

\subsection{\textcolor{Chapter }{BlockDiagonalRepresentation (for IsGroupHomomorphism)}}
\logpage{[ 1, 1, 2 ]}\nobreak
\hyperdef{L}{X85D284E386A4257F}{}
{\noindent\textcolor{FuncColor}{$\triangleright$\enspace\texttt{BlockDiagonalRepresentation({\mdseries\slshape rho})\index{BlockDiagonalRepresentation@\texttt{BlockDiagonalRepresentation}!for IsGroupHomomorphism}
\label{BlockDiagonalRepresentation:for IsGroupHomomorphism}
}\hfill{\scriptsize (attribute)}}\\
\textbf{\indent Returns:\ }
Representation of $G$ isomorphic to $\rho$ where the images $\rho(g)$ are block diagonalized. 



 This is just a convenience operation that uses \texttt{BlockDiagonalBasis} (\ref{BlockDiagonalBasis}) to calculate the basis change matrix and put $\rho$ into a nice form. }

 }

 }

   
\chapter{\textcolor{Chapter }{Calculating centralizer rings}}\label{Chapter_Calculating_centralizer_rings}
\logpage{[ 2, 0, 0 ]}
\hyperdef{L}{X78E835B779A7EB23}{}
{
  
\section{\textcolor{Chapter }{Centralizer (commutant) of a representation}}\label{Chapter_Calculating_centralizer_rings_Section_Centralizer_commutant_of_a_representation}
\logpage{[ 2, 1, 0 ]}
\hyperdef{L}{X823F06E685D5D655}{}
{
  

\subsection{\textcolor{Chapter }{RepresentationCentralizerBlocks}}
\logpage{[ 2, 1, 1 ]}\nobreak
\hyperdef{L}{X798AD78F858A73FF}{}
{\noindent\textcolor{FuncColor}{$\triangleright$\enspace\texttt{RepresentationCentralizerBlocks({\mdseries\slshape rho})\index{RepresentationCentralizerBlocks@\texttt{RepresentationCentralizerBlocks}}
\label{RepresentationCentralizerBlocks}
}\hfill{\scriptsize (function)}}\\
\textbf{\indent Returns:\ }
List of standard generators (as a vector space) for the centralizer ring of $\rho(G)$, written in the basis given by \texttt{BlockDiagonalBasis} (\ref{BlockDiagonalBasis}). The matrices are given as a list of blocks. 



 Let $G$ have irreducible representations $\rho_i$ with multiplicities $m_i$. The centralizer has dimension $\sum_i m_i^2$ as a $\mathbb{C}$-vector space. This function gives the minimal number of generators required. }

 }

 
\section{\textcolor{Chapter }{Useful convenience functions}}\label{Chapter_Calculating_centralizer_rings_Section_Useful_convenience_functions}
\logpage{[ 2, 2, 0 ]}
\hyperdef{L}{X7E993D5E86E2047A}{}
{
  

\subsection{\textcolor{Chapter }{RepresentationCentralizer}}
\logpage{[ 2, 2, 1 ]}\nobreak
\hyperdef{L}{X7DEC18DE7E169D9D}{}
{\noindent\textcolor{FuncColor}{$\triangleright$\enspace\texttt{RepresentationCentralizer({\mdseries\slshape rho})\index{RepresentationCentralizer@\texttt{RepresentationCentralizer}}
\label{RepresentationCentralizer}
}\hfill{\scriptsize (function)}}\\
\textbf{\indent Returns:\ }
List of standard generators (as a vector space) for the centralizer ring of $\rho(G)$. 



 This gives the same result as \texttt{RepresentationCentralizerBlocks} (\ref{RepresentationCentralizerBlocks}), but with the matrices given in their entirety: not as lists of blocks, but
as full matrices. }

 

\subsection{\textcolor{Chapter }{RepresentationCentralizerDecomposed}}
\logpage{[ 2, 2, 2 ]}\nobreak
\hyperdef{L}{X7DB7E8257A2BFEC9}{}
{\noindent\textcolor{FuncColor}{$\triangleright$\enspace\texttt{RepresentationCentralizerDecomposed({\mdseries\slshape rho})\index{RepresentationCentralizerDecomposed@\texttt{RepresentationCentralizerDecomposed}}
\label{RepresentationCentralizerDecomposed}
}\hfill{\scriptsize (function)}}\\
\textbf{\indent Returns:\ }
List of generators (as a vector space) for the centralizer ring of $\rho(G)$, under the map taking each identity matrix block to a 1 by 1 block. 



 This function is here to demonstrate the reduction in dimension of the
centralizer $C$ by writing it in the basis given by \texttt{BlockDiagonalBasis} (\ref{BlockDiagonalBasis}). The matrices given are as reduced as possible. }

 }

 }

   
\chapter{\textcolor{Chapter }{Useful predicates}}\label{Chapter_Useful_predicates}
\logpage{[ 3, 0, 0 ]}
\hyperdef{L}{X788057358673E2F3}{}
{
  
\section{\textcolor{Chapter }{Types of group representations}}\label{Chapter_Useful_predicates_Section_Types_of_group_representations}
\logpage{[ 3, 1, 0 ]}
\hyperdef{L}{X7E2261C8864B1709}{}
{
  

\subsection{\textcolor{Chapter }{IsFiniteGroupLinearRepresentation (for IsGroupHomomorphism)}}
\logpage{[ 3, 1, 1 ]}\nobreak
\hyperdef{L}{X8631A1417C3C1D88}{}
{\noindent\textcolor{FuncColor}{$\triangleright$\enspace\texttt{IsFiniteGroupLinearRepresentation({\mdseries\slshape rho})\index{IsFiniteGroupLinearRepresentation@\texttt{IsFiniteGroupLinearRepresentation}!for IsGroupHomomorphism}
\label{IsFiniteGroupLinearRepresentation:for IsGroupHomomorphism}
}\hfill{\scriptsize (attribute)}}\\
\textbf{\indent Returns:\ }
true or false 



 Tells you if \mbox{\texttt{\mdseries\slshape rho}} is a linear representation of a finite group. This is important since Serre's
algorithms only work on these. }

 

\subsection{\textcolor{Chapter }{IsFiniteGroupPermutationRepresentation (for IsGroupHomomorphism)}}
\logpage{[ 3, 1, 2 ]}\nobreak
\hyperdef{L}{X826D5ADF7FA87782}{}
{\noindent\textcolor{FuncColor}{$\triangleright$\enspace\texttt{IsFiniteGroupPermutationRepresentation({\mdseries\slshape rho})\index{IsFiniteGroupPermutationRepresentation@\texttt{IsFinite}\-\texttt{Group}\-\texttt{Permutation}\-\texttt{Representation}!for IsGroupHomomorphism}
\label{IsFiniteGroupPermutationRepresentation:for IsGroupHomomorphism}
}\hfill{\scriptsize (attribute)}}\\
\textbf{\indent Returns:\ }
true or false 



 Tells you if \mbox{\texttt{\mdseries\slshape rho}} is a homomorphism from finite group to a permutation group. Such homomorphisms
occur often in applications. }

 }

 }

   
\chapter{\textcolor{Chapter }{Computing decompositions of representations}}\label{Chapter_Computing_decompositions_of_representations}
\logpage{[ 4, 0, 0 ]}
\hyperdef{L}{X7F968DF987DE4A6E}{}
{
  
\section{\textcolor{Chapter }{Algorithms due to Serre}}\label{Chapter_Computing_decompositions_of_representations_Section_Algorithms_due_to_Serre}
\logpage{[ 4, 1, 0 ]}
\hyperdef{L}{X7C22F13E80A74438}{}
{
  These operations compute various decompositions of a representation $\rho : G \to GL(V)$ where $G$ is finite and $V$ is a finite-dimensional $\mathbb{C}$-vector space. The terms used here are taken from Serre's Linear
Representations of Finite Groups. 

\subsection{\textcolor{Chapter }{CanonicalDecomposition (for IsGroupHomomorphism)}}
\logpage{[ 4, 1, 1 ]}\nobreak
\hyperdef{L}{X8316BC7684A911AE}{}
{\noindent\textcolor{FuncColor}{$\triangleright$\enspace\texttt{CanonicalDecomposition({\mdseries\slshape rho})\index{CanonicalDecomposition@\texttt{CanonicalDecomposition}!for IsGroupHomomorphism}
\label{CanonicalDecomposition:for IsGroupHomomorphism}
}\hfill{\scriptsize (attribute)}}\\
\textbf{\indent Returns:\ }
List of vector spaces $V_i$, each $G$-invariant and a direct sum of isomorphic irreducibles. That is, for each $i$, $V_i \cong \oplus_j W_i$ (as representations) where $W_i$ is an irreducible $G$-invariant vector space. 



 Computes the canonical decomposition of $V$ into $\oplus_i\;V_i$ using the formulas for projections $V \to V_i$ due to Serre. }

 

\subsection{\textcolor{Chapter }{IrreducibleDecomposition (for IsGroupHomomorphism)}}
\logpage{[ 4, 1, 2 ]}\nobreak
\hyperdef{L}{X7DBB32527BDB15B3}{}
{\noindent\textcolor{FuncColor}{$\triangleright$\enspace\texttt{IrreducibleDecomposition({\mdseries\slshape rho})\index{IrreducibleDecomposition@\texttt{IrreducibleDecomposition}!for IsGroupHomomorphism}
\label{IrreducibleDecomposition:for IsGroupHomomorphism}
}\hfill{\scriptsize (attribute)}}\\
\textbf{\indent Returns:\ }
List of vector spaces $W_j$ such that $V = \oplus_j W_j$ and each $W_j$ is an irreducible $G$-invariant vector space. 



 Computes the decomposition of $V$ into irreducible subprepresentations. }

 

\subsection{\textcolor{Chapter }{IrreducibleDecompositionCollected (for IsGroupHomomorphism)}}
\logpage{[ 4, 1, 3 ]}\nobreak
\hyperdef{L}{X7FA264B77B938090}{}
{\noindent\textcolor{FuncColor}{$\triangleright$\enspace\texttt{IrreducibleDecompositionCollected({\mdseries\slshape rho})\index{IrreducibleDecompositionCollected@\texttt{IrreducibleDecompositionCollected}!for IsGroupHomomorphism}
\label{IrreducibleDecompositionCollected:for IsGroupHomomorphism}
}\hfill{\scriptsize (attribute)}}\\
\textbf{\indent Returns:\ }
List of lists $V_i$ of vector spaces $V_{ij}$ such that $V = \oplus_i \oplus_j V_{ij}$ and $V_{ik} \cong V_{il}$ for all $i$, $k$ and $l$ (as representations). 



 Computes the decomposition of $V$ into irreducible subrepresentations, grouping together the isomorphic
subrepresentations. }

 }

 }

 \def\indexname{Index\logpage{[ "Ind", 0, 0 ]}
\hyperdef{L}{X83A0356F839C696F}{}
}

\cleardoublepage
\phantomsection
\addcontentsline{toc}{chapter}{Index}


\printindex

\immediate\write\pagenrlog{["Ind", 0, 0], \arabic{page},}
\newpage
\immediate\write\pagenrlog{["End"], \arabic{page}];}
\immediate\closeout\pagenrlog
\end{document}
